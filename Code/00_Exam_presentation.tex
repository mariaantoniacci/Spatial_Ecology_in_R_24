\documentclass{beamer}
\usepackage{graphicx} % Required for inserting images
\usepackage{hyperref} % per i collegamenti ipertestuali

\usetheme{Frankfurt} % stile della presentazione
\usecolortheme{spruce} % colore

\title{Wildfires in Evros region, Greece}
\author{Maria Antoniacci}
\institute{Scienze e Gestione della Natura - Unibo}
%\date{}

\begin{document}

\maketitle

\AtBeginSection[]
{
\begin{frame}{Outline}
\tableofcontents[currentsection]
\end{frame}
} % ad ogni sezione nuova mi fa vedere l'outline

\section{Introduction} % 1° sezione

\begin{frame}{2023 Wildfires in Greece}

\begin{columns}

    \begin{column}{0.65\textwidth}
    \begin{figure}
    \centering
    \includegraphics[width=0.6\linewidth]{map.jpg}
    \end{figure}
    
     \begin{figure}
     \centering
     \includegraphics[width=0.6\linewidth]{ansa.jpg}
    
    {\tiny{Source: Sentinel-2, processed by ESA, 2023.}}
   \end{figure}  
    
    \end{column}

    \begin{column}{0.5\textwidth}
        \scriptsize In \textbf{August 2023}, a massive \textbf{wildfire}\\ broke out in the \textbf{Evros} region of northeastern Greece, reaching Alexandroupolis.
        
        \bigskip
        Approximately \textbf{80,000 hectares}\\ of land were burned, including\\ the forest within the Dadia -\\ Lefkimi - Soufli Forest National Park.
        
        \bigskip
        This event is considered to be the\\ largest wildfire ever recorded in \\ Europe since 2000. 
       
    \end{column}
    
\end{columns}
\end{frame}

\section{Objectives} % 2° sezione

\begin{frame}{Objectives}
    The objectives of the analysis are:
    \bigskip
    \begin{itemize}
    \item    1: Assessing \textbf{Pre- and Post- Wildfire} Conditions using the Normalized Burn Ratio (\textbf{NBR} and \textbf{dNBR})

 \bigskip
    
    \item   2: Evaluating \textbf{Vegetation Recovery Trends}  over the years in June using the Normalized Difference Vegetation Index (\textbf{NDVI}) 
    \end{itemize}
\end{frame}

\section{Materials and methods} % 3° sezione
\begin{frame}{Data collection}
    Sentinel-2 satellite images were taken from 
    \textbf{\href{https://browser.dataspace.copernicus.eu}{Copernicus Browser}}
    \bigskip
    \begin{itemize}
        \item  Selecting area of interest
        \item  Filtering with cloud coverage lower than 10\%
        \item Downloading single bands B4, B3, B2 for True color, B8 (NIR), B12 (SWIR)
        \item Downloading in \textit{.tiff}
    \end{itemize}
\end{frame}

\begin{frame}{Packages} % titolo in cima
    The analyses require the following packages:
    \bigskip % per lasciare spazio
    \begin{itemize} % elenco puntato
        \item terra
        \item imageRy
        \item viridis
        \item ggplot2
        \item patchwork
    \end{itemize}
\end{frame}

\begin{frame}{Main functions}
\bigskip
\begin{columns}
    \begin{column}{0.5\textwidth}
    \begin{itemize}
        \item    \texttt{library()} 
        \item    \texttt{setwd()} 
        \item    \texttt{rast()} 
        \item    \texttt{im.plotRGB()}
        \item    \texttt{plot()}
        \item    \texttt{par()}
        \item    \texttt{viridis()}
    \end{itemize}
    \end{column}
    \begin{column}{0.5\textwidth}  
     \begin{itemize}
        \item    \texttt{im.classify()}
        \item    \texttt{freq()} 
        \item    \texttt{ncell()}
        \item    \texttt{data.frame()}
        \item    \texttt{ggplot()}
        \item    \texttt{patchwork()}
        \item    \texttt{focal()}
        \end{itemize}
    \end{column}
\end{columns}
\end{frame}

\section{Analysis - 1} 
\begin{frame}{Before and After the Wildfire}
\centering
Assessing Pre- and Post- Wildfire conditions in southern Evros. 

\bigskip

\centering True Colors

\begin{figure}
    \centering
    \includegraphics[width=1\linewidth]{TC_PRE_POST.jpeg}
\end{figure}
\end{frame}


\begin{frame}{Spectral Indices -  SWIR}
SWIR (Short-Wave InfraRed) shows \textbf{high reflectance on burned vegetation} and low reflectance of healthy vegetation.

 \bigskip
 
 \begin{figure}
    \centering
    \includegraphics[width=1\linewidth]{SWIR_PRE_POST.jpeg}
   \end{figure}
\end{frame}

\begin{frame}{Spectral Indices - NBR}
 To analyze images \textbf{after wildfires} NBR index is used.

\bigskip
 \textbf{Normalized Burn Ratio} (NBR) is a normalized index that uses SWIR and NIR bands.

 \begin{figure}
    \centering
    \includegraphics[width=1\linewidth]{NBR_PRE_POST.jpeg}
   \end{figure}
\end{frame}

\begin{frame}{Spectral Indices - dNBR}
\textbf{dNBR} is used to assess the severity of the burn. It is the difference between NBR before and after the fire.

\bigskip
Higher positive dNBR values indicate greater \textbf{burn severity}.

\begin{center}
     \centering
     \includegraphics[width=0.8\linewidth]{dNBR1.jpeg}
 \end{center} 
\end{frame}

\begin{frame}{Wildfire Damage Classification}
 \centering Identifying levels of damage in Evros area based on dNBR:
 
 \begin{columns}
    \begin{column}[t]{0.30\textwidth}
       \centering
       \bigskip
       \bigskip
       \bigskip
        \begin{itemize}
         \item Severely damaged=13\%
         \item Moderately damaged=12\%
        \end{itemize}
    \end{column} 

    \begin{column}[t]{0.80\textwidth}
            \centering
            \begin{figure}
            \includegraphics[width=1\linewidth]{dNBR_CLASSIFICATION.jpeg}  
            \end{figure}
    \end{column} 
\end{columns}
\end{frame}

\section{Analysis - 2} 
\begin{frame}{Vegetation Recovery Trends}
 \bigskip
 \centering Evaluating Vegetation Dynamics over the years 
\begin{figure}
            \includegraphics[width=1\linewidth]{TC_23_24_25.jpeg}  
            \end{figure}
\end{frame}

\begin{frame}{Spectral Indices - NDVI}
\textbf{Normalized Difference Vegetation Index} (NDVI) is used here to assess how vegetation responds to disturbance caused by fire over time.

\centering 
    \begin{equation*}
        NDVI = \frac{NIR - RED}{NIR + RED}
    \end{equation*}
    \begin{figure}
            \includegraphics[width=1\linewidth]{NDVI_COMP.jpeg}  
    \end{figure}
\end{frame}

\begin{frame}{NDVI-based Classification}
\bigskip
NDVI-based classification helps visualize vegetation dynamics over three years by showing variations between damaged and healthy areas. 
\begin{figure}
        \includegraphics[width=1\linewidth]{CLASS_COMP232425.jpeg}  
    \end{figure}
\end{frame}

\begin{frame}{NDVI-based Classification}

\begin{columns}

    \begin{column}{0.6\textwidth}
    
        \begin{figure}
        \centering
                \includegraphics[width=1\linewidth]{ClassJUNE.jpeg}  
        \end{figure}
    \end{column}

    \begin{column}{0.4\textwidth}
    Percentages of vegetation in good condition over the years: \begin{itemize}
            \item June 2023: 71\%
            \item June 2024: 44\%
            \item June 2025: 59\%
        \end{itemize}
        \bigskip
    The data suggest that \textbf{vegetation is slowly recovering}.
    \end{column}

\end{columns}
\end{frame}

\begin{frame}{Measuring Spatial Variability on NDVI}
\centering
Moving Window \textbf{3x3}

\begin{figure}
        \centering
        \includegraphics[width=1\linewidth]{SD_3X3.jpeg}
    \end{figure}  
\end{frame}

\begin{frame}{Measuring Spatial Variability on NDVI}
\centering
Moving Window \textbf{7x7} 

\begin{figure}
        \centering
        \includegraphics[width=1\linewidth]{SD_7X7.jpeg}
    \end{figure}   
\end{frame}


\section{Conclusions}

\begin{frame}{Considerations}
To summarize the result of the analyses:
\bigskip
\begin{itemize} 
\item Approximately 25\% of the study area in southern Evros was moderately to severely damaged by the wildfire
    \bigskip
    \pause \item NBR and dNBR map burn severity effectively
    \bigskip
     \pause \item NDVI shows a decline in 2024, followed by a partial recovery of vegetation in 2025
    \bigskip
     \pause \item Local Standard Deviation Measure reveals differences in spatial variability between 2024 and 2025
   
\end{itemize}
\end{frame}

\begin{frame}
\centering
\large {Thank you for your attention!}
\end{frame}


\end{document}
